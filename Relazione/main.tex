\documentclass[letterpaper,12pt]{article}
\usepackage{tabularx} % extra features for tabular environment
\usepackage{amsmath}  % improve math presentation
\usepackage{graphicx} % takes care of graphic including machinery
\usepackage[margin=1in,letterpaper]{geometry} % decreases margins
\usepackage{cite} % takes care of citations
\usepackage[final]{hyperref} % adds hyper links inside the generated pdf file
\hypersetup{
	colorlinks=true,       % false: boxed links; true: colored links
	linkcolor=blue,        % color of internal links
	citecolor=blue,        % color of links to bibliography
	filecolor=magenta,     % color of file links
	urlcolor=blue         
}
\usepackage{blindtext}
%++++++++++++++++++++++++++++++++++++++++


\begin{document}

\title{Relazione progetto machine learning}
\author{Martino Pettinari 866496, Davide Creati 869274}
\maketitle

\begin{abstract}
Nella relazione abbiamo cercato di riassumere tutti gli step che ci hanno permesso di arrivare all'obbiettivo del progetto ovvero addestrare 2 modelli scelti e valutarne la correttezza e precisione. Nel processo si includono scelte stilistiche, assunzioni o ipotesi fatte durante lo svolgimento.
\end{abstract}


\section{Introduzione e obiettivi}

Per la realizzazione di questo progetto, ci siamo concentrati nel lavorare con dati provenienti dal mondo del tennis. La nostra scelta di concentrarci su questo sport è stata guidata principalmente da una ricerca effettuata. Dall'analisi è emerso che, tra l'infinità degli sport che vengono praticati al giorno d'oggi, il tennis offro ottime possibilità per il mondo del machine learning. Ciò è dato dal fatto che è composto da un ambiente meno complesso, un esempio è dato dal fatto che ogni match viene giocato da soli 2 giocatori che si sfidano (dato che nel nostro caso abbiamo considerato solo match singoli e non in doppio). 

Il secondo passaggio, una volta identificato il macro perimetro, è stata identificare dei tornei che ci permettessero di avere dati recenti e aggiornati. Per fare ciò abbiamo deciso di analizzare e seguire i tornei del circuito professionistico ATP per il quale, ogni anni, vengono giocati 4 Grand Slam. Il circuito professionistico ATP è composto da molteplici tornei che si svolgono durante l'arco dell'anno. Ognuno di questi tornei da punteggi diversi, in base all'importanza del torneo. 

Per fini progettuali ci siamo focalizzati solo sui tornei Grand Slam (ovvero quelli con il maggior punteggio) avvenuti nel 2023. Questa decisione è stata guidata dal fatto che siano eventi in cui i datti hanno una maggiore consistenza. Questo poiché i giocatori offrano prestazioni al massimo delle loro capacità dato che, la posta in gioco (rappresentata dai punti assegnati), crea un incentivo per i giocatori, garantendo partite di alta intensità e competizione. 

L'obiettivo centrale di questo progetto è stato sviluppare 2 modelli scelti di modelli di machine learning in grado di prevedere l'esito di ogni singola partita disputata durante l'ultimo Grand Slam del 2023, utilizzando i dati degli altri 3 tornei per l'addestramento. 

\section{Dati utilizzati}

\end{document}
